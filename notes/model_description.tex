% Created 2022-01-28 sex 14:51
% Intended LaTeX compiler: pdflatex
\documentclass[11pt]{article}
\usepackage[utf8]{inputenc}
\usepackage[T1]{fontenc}
\usepackage{graphicx}
\usepackage{longtable}
\usepackage{wrapfig}
\usepackage{rotating}
\usepackage[normalem]{ulem}
\usepackage{amsmath}
\usepackage{amssymb}
\usepackage{capt-of}
\usepackage{hyperref}
\author{Marcelo Veloso Maciel}
\date{\today}
\title{Model Description}
\hypersetup{
 pdfauthor={Marcelo Veloso Maciel},
 pdftitle={Model Description},
 pdfkeywords={},
 pdfsubject={},
 pdfcreator={Emacs 27.2 (Org mode 9.6)},
 pdflang={English}}
\usepackage{biblatex}

\begin{document}

\maketitle
\tableofcontents

\section{Glossary}
\begin{itemize}
  \item \(\rho\): radius parties use to sample their candidates;
  \item \(\kappa\): benefit of the doubt parameter.
  \item \(\omega\): weight given to the previous supporters vs new supporters.
\end{itemize}


\section{Initial Condition}
\label{sec:orga7f82a7}
\begin{enumerate}
  \item Sample half of voters positions from Normal(43.25,10) and another half
        from Normal(56.75, 10). I'm using Cohen's D to calculate these overlapping Normals (\url{https://rpsychologist.com/cohend/});
\item Sample parties positions from the previously sampled positions.
\end{enumerate}

\section{Stepping}
There are three kinds of steps.

\begin{itemize}
  \item In the first step:
  \begin{enumerate}
    \item  Each party picks a random candidate within a radius \(\rho\) from their current position.
    \item Every voter votes to the candidate that is closest to them;
    \item Every voter adds the party they voted for  to their vote\_ballot\_tracker
  \end{enumerate}
  \item In the second and third steps:
  \begin{enumerate}
    \item Each party picks 4 random candidates within a radius \(\rho\) from their current position;
    \item Voters that gave support in the previous election to this party vote in the primaries (plurality) to choose the party candidate.
    \item Every voter votes to the candidate that is closest to them and add the party they voted  to their vote\_ballot\_tracker
  \end{enumerate}
  \item From step 4 onwards:

  \begin{enumerate}
    \item (ONLY ON STEP 4): voter randomly assumes a party id based on their vote\_ballot\_tracker: if the agent voted for the party for 3/4 of the steps, it has a probability of 3/4 of being chosen as the agent party id.

\item Sample 4 candidates for each party according to a radius \(\delta\)
\begin{itemize}
\item With the conditional that the candidates must have the party id of the party;
\item right now, \(\delta\) is a global parameter.
\end{itemize}
\item Each party chooses a candidate according to a voting procedure among its supporters

\begin{itemize}
\item Right now it is plurality. Runoff is already working too.
\end{itemize}
\item Voters vote according to the global benefit of the doubt parameter, \(\kappa\)
\begin{itemize}
\item if euclidean-distance(position-candidate-closest-to-me, position-my-party-candidate) > \(\kappa\) then vote for the candidate-closest-to-me else vote for my-party-candidate
\item Right now there are two versions of  \(\kappa\):
\begin{itemize}
\item global parameter;
\item baseline global \(\kappa\) * proportion of times agent voted for that party (updated at each time step)
\end{itemize}
\end{itemize}
    \item The candidate with most votes wins

    \item Voters change their id according to their party loyalty, which is the
          probability of keeping their party id.

\begin{itemize}
  \item If the agent voted against its party id, then keep\_party\_id\_prob =
        proportion\_IvotedForThisParty * (1-proportion\_peers\_like\_me);
  \item Else: keep\_party\_id\_prob = proportion\_IvotedForThisParty * proportion\_peers\_like\_me;
  \item As usual  vote\_ballot\_tracker is updated.
\end{itemize}

\item Parties change their position
\begin{itemize}
  \item \(\omega\)*mean-previous-supporters + (1- \(\omega\))*mean-new-supporters
  \item \(\omega\) may be a global parameter or something that parties have as an
        internal variable;
\end{itemize}

\end{enumerate}
\end{itemize}

\section{Data Collection}

Collect system's measures: Distance between parties and party loyalty.



\end{document}
